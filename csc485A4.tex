\documentclass[11pt]{article} 
\begin{document}

\title{CSC485 - A4}
\author{Michael Dzamba - 994803806}

\maketitle \tiny
NOTE: I am sorry that the formatting is a bit off and that I used probably two more pages then I needed to. I will try to make it up to the planet somehow. \normalsize

\section{Part 1: Finding hypernym relations}
\subsection{Please see end of assignment for code}
\subsection{Chunker}
The chunker that was used is a modified version of the original chunker. A modification was added to the original NP chunker that allows NP expressions to be of the form, NP (PP NP)*. This helps prevent the chunker from picking up only the last NP in a string of NPs used as prepositional attachments, for use with a predefined hypernym relations. For example, with the original chunker the following would be chunked,\\ \\
\textit{'(NP Dogs) (IN from) (NP Russia) such as (NP Nikola)'} \\ \\
And after using the 'such as' rule from Marti Hearst's paper we arrive at the following relation,\\ \\
\textit{Hyponym(Nikola,Russia)} \\ \\
The modified chunker used would chunk the sentence as, \\ \\
\textit{'(NP Dogs from Russia) such as (NP Nikola)'} \\ \\
Allowing for post processing steps later on to detect the propositional phrase in the first NP and drop the sentence, because of ambiguity of which word(s) the relation is being made for.
\subsection{Lexicosyntactic patterns}
The implemented Lexicosyntactic patterns are the 6 from Marti Heart's paper and an additional 4 that were mentioned in Jurafsky and Martin from Snow et. al (2005) paper. The last 4 patterns are implemented but commented out, because they produce a large fraction of additional instances, of which many are incorrect.
\subsection{Evaluation and report}
\begin{tabular}{c c c c c c}
Confidence level & Instances found & Case 1 & Case 2 & Case 3 & Case 4 \\ \hline
Low & 336 & 24 & 6 & 86 & 224 \\
Medium & 2 & 1 & 0 & 0 & 1\\
High & 0 & 0 & 0 & 0 & 0 
\end{tabular} \\ \\ \\
There were only two medium confidence relations found, these are, \\ \\
\textit{hyponym(demons, spiritual beings )} \\
This relation is found in case 4, it is there because 'spiritual beings' has no synset in wordnet. Although this relation is perfectly fine according to me. \\ \\ 
\textit{hyponym(bacteria, microorganisms)}\\
This relation is found in case 1, and is correct.


\subsubsection{Suggestions belonging to case 1}
\begin{minipage}[b]{0.5\linewidth}\centering
\begin{tabular}{c c}
Hypernym & Hyponym \\ \hline
attributes & caste \\
subjects & architecture \\
metals & iron \\
condition & dirt \\
substances & greases \\
gases & oxygen \\
organizations & newspapers \\
substances & particles \\
substances & oils \\
jewels & emeralds \\
object & tree \\
meats & hamburger \\
buildings & dormitories \\
\end{tabular}
\end{minipage}
\hspace{0.5cm}
\begin{minipage}[b]{0.5\linewidth}
\centering
\begin{tabular}{c c}
Hypernym & Hyponym \\ \hline
attributes & sex \\
records & documents \\
language & russian \\
animals & mink \\
jewels & pearls \\
piping & sewers \\
buildings & hotels \\
buildings & houses \\
records & memoranda \\
city & atlanta \\
liquids & water \\
microorganisms & bacteria \\
\end{tabular}
\end{minipage}


\subsubsection{Suggestions belonging to case 2}
\begin{minipage}[b]{0.5\linewidth}\centering
\begin{tabular}{c c}
Hypernym & Hyponym \\ \hline
developments & processes \\
products & chemicals \\
developments & uses \\
\end{tabular}
\end{minipage}
\hspace{0.5cm}
\begin{minipage}[b]{0.5\linewidth}
\centering
\begin{tabular}{c c}
Hypernym & Hyponym \\ \hline
submarines & ships \\
records & documents \\
fertilizers & materials \\
\end{tabular}
\end{minipage}


\subsubsection{Suggestions belonging to case 3}
\begin{tabular}{c c c c}
Hypernym & Hyponym & Correct & Comments \\ \hline
diseases & encephalitis & Yes \\
industry & irrigation & No & Sentence wrongly chunked\\
spots & knees & Yes & Very general\\
ologies & parisology & Yes\\
developments & processes & Yes \\
nations & peoples & No\\
members & hands & Yes  \\
concepts & blasphemy & Yes & Very general\\
compensation & salary & Yes \\
equipment & dies & No & Dies(dyes) spelt wrong\\
products & chemicals & Yes & Very general\\
developments & uses & Yes \\
instructions & time & No \\
concept & suggestibility &  Yes & Very general\\
concepts & damnation & Yes & Very general\\
ologies & amphibology & Yes \\
subjects & textiles & No \\
managers & capitalists & Yes\\
provisions & terms & No\\
foliage & fern & Yes\\
businesses & janitors & No \\
submarines & ships & No & Relation backwards\\
species & dog & Yes\\
things & grillwork & Yes & Very general\\
condition & flies & No\\
gases & purposes & No & Badly chunked\\
members & feet & Yes\\
areas & education & Yes & Very general\\
industry & transport & Yes \\
concepts & chain & No & Badly chunked\\
piping & water & No\\
developments & products & Yes & Very general\\
industry & power & Yes\\
girls & sand & No\\
reason & economy & No\\
nation & switzerland & Yes\\
\end{tabular}\\

\begin{tabular}{c c c c}
Hypernym & Hyponym & Correct & Comments \\ \hline
subjects & color & No\\
passage & following & No\\
businesses & factories & Yes\\
microorganisms & molds & Yes\\
paraphernalia & vehicles & Yes & Very general\\
people & spouses & Yes\\
concepts & purgation & Yes & Very general\\
corporations & banks & Yes\\
pictures & durer & No\\
property & facilities & No\\
members & head & Yes\\
records & correspondence & No\\
additives & polyester & Yes & Very general \\
\end{tabular}
\\ \\ \\
The above 50 examples from case 3 show 19 examples where the relation is regarded as incorrect and 31 where the relation is regarded as correct. Some of the short comings of my program are the ability to parse more complicated sentence structures. For example \textit{industry} is not a hypernym of \textit{irrigation}, but this is a bit misleading to the original sentence,\\ \\
\small
\textit{ (S  (NP Industry/NN)  ,/,  including/IN  (NP the/AT production/NN of/IN fertilizer/NN)  ,/,  (NP irrigation/NN)  and/CC  (NP power/NN)  ,/,  (NP transport/NN)  and/CC  (NP communications/NNS)  ,/,  and/CC  (NP credit/NN institutions/NNS)  ;/.  ;/.) }\\ \\ \normalsize
I think the sentence is trying to say that \textit{industry} is a hypernym of \textit{the production of irrigation}, which would be a correct statement.\\
Another short coming is the very general statements made, that although may be considered a hypernym/hyponym relation, they are not all that useful. For example, \\ \\
The hyponym \textit{education} of hypernym \textit{areas}. Knowing that something is a hyponym of \textit{areas} gives very little information, because the set of hyponyms of \textit{areas} is so large.\\ \\
This is not a short coming of the program though, but it is a problem with the relation at hand. \\
Many successes of the program can be seen even though the relation in wordnet is not present, for example,\\ \\
The hyponym \textit{encephalitis} of hypernym \textit{diseases}\\

\subsubsection{Suggestions belonging to case 4}
\begin{tabular}{c c c c}
Hypernym & Hyponym & Correct & Comments \\ \hline
large sign manufacturers & advance neon sign co & Yes & Very specific \\
system & fromms & No & Fromm's is a person\\
societies & american historical association & Yes & Very specific \\
marine life & clams & Yes\\
900calorie milk formulas & quaker oatss quota & Yes & Very specific\\
necessities & own clothing & Yes\\
mighty hand & master & ? & Unclear\\
promotional material & posters & Yes\\
taurida & crimean peninsula & No & Was part of Taurida\\
work & new jazz experiments & No\\
pathogenic phenomena & undue regression & Yes \\
relevant matters & such sale & No\\
intellectuals & aaron copland & Yes & Very specific\\
radiationproduced diseases & leukemia & Yes & Bad parse\\
organ meats & chicken liver & Yes\\
emotional disturbance & frustration & Yes \\
thing & mental illness & Yes & Very general\\
good cause & illness & Yes\\
companies & leesona & Yes & Very specific\\
heavy material & books & Yes\\
hard material & metal & Yes\\
organizations & radiotv stations & Yes & Bad parse\\
promotional material & brochures & Yes\\
public facilities & roads & Yes\\
policy literature & national security council papers & Yes & Very specific\\
poor white wine & dry wine & ? & Relative to person\\
small businesses & wholesalers & Yes \\
cholesterolrich foods & even eggs &  Yes & Bad parse\\
principle & cousin elec & Yes & Very specific\\
paraphernalia & architectural developments & Yes & Very general \\
federal land & national forest & Yes \\
small object & cinder & Yes\\
professional organizations & various trade associations & Yes  \\
employment opportunities & regular job & Yes\\
diplomats & israeli foreign minister & Yes\\
ten largest cities & new york & Yes & Bad parse\\
such ladies accessories & pistols & No\\
\end{tabular}\\ 

\begin{tabular}{c c c c}
Hypernym & Hyponym & Correct & Comments \\ \hline
st petersburg & joe dimaggio & No & Bad parse\\
25 minutes & san antonio & No & Bad parse\\
oil mill supplies & belting & Yes\\
groups & kiwanis & Yes \\
shiny characters & harry truman & Yes & Very specific\\
region & coastal lowlands & Yes\\
organ meats & beef  & Yes\\
such ladies accessories & automatic weapons & No\\
percussive instruments & drums & Yes \\
societies & american folklore society & Yes \\
civic organizations & local banks & Yes\\
sober commentators & nation & No\\
reports & publicopinion polls & Yes & Bad parse \\
\end{tabular} \\ \\ \\
The above 50 examples from case 4 show 9 incorrect relations, 2 unclear relations and 39 correct relations. Although a relation may be classified as correct, this does not mean that the relation is useful in a information sense (as described above under case 3 examples). One of the short comings of my program is its ability to parse more complex sentence structures. For example, \\ \\ \small
\textit{(S  From/IN  (NP    International/JJ-TL    Airport/NN-TL    in/IN    Los/NP    Angeles/NP    to/IN    International/JJ-TL    Airport/NN-TL    in/IN    Houston/NP)  ,/,  as/CS  (NP the/AT great/JJ four-jet/JJ Boeing/NP-TL)  707/CD-TL  flies/VBZ  ,/,  is/BEZ  (NP a/AT routine/JJ five/CD hours/NNS)  and/CC  (NP 25/CD minutes/NNS)  ,/,  including/IN  (NP stopovers/NNS at/IN Phoenix/NP)  ,/,  (NP El/NP Paso/NP)  ,/,  and/CC  (NP San/NP Antonio/NP)  ./.) } \\ \\ \normalsize
Is parsed with one of the lexicosyntactic patterns to produce the relation hyponym(san antonio,25 minutes). This is due to the fact that \textit{(NP 25/CD minutes/NNS)} is not chunked together with its larger constituent, and is used as if it were the sentence, \\ \\ \small
\textit{25 minutes, including San antonio}\\ \\ \normalsize
Another short coming can be seen in the sentence,\\ \\ \small
\textit{(S  (NP Ralph/NP Houk/NP)  ,/,  (NP    successor/NN    to/IN    Casey/NP    Stengel/NP    at/IN    the/AT    Yankee/NP    helm/NN)  ,/,  plans/VBZ  to/TO  bring/VB  (NP the/AT entire/JJ New/JJ-TL York/NP-TL squad/NN)  here/RB  from/IN  (NP St./NP Petersburg/NP)  ,/,  including/IN  (NP Joe/NP Dimaggio/NP)  and/CC  (NP large/JJ crowds/NNS)  are/BER  anticipated/VBN  for/IN  both/ABX  (NP weekend/NN games/NNS)  ./.) }\\ \\ \normalsize
Which gives rise to the relation hyponym(joe dimaggio, st petersburg), when it should really be something like hyponym(joe dimaggio, new york squad member). This is most likely the chunkers fault as it did not chunk \textit{(NP St./NP Petersburg/NP)} with \textit{(NP the/AT entire/JJ New/JJ-TL York/NP-TL squad/NN)} and leave it for additional post processing. Doing this can get tricky though, since propositional phrase attachment is not always straightforward.\\
Success can be seen from the fact that from the above 50 examples, 39 are regarded by me to be correct (not necessarily useful).\\




\section{Part 2: Find causal relations}
\subsection{Please see end of assignment for code}
\begin{tabular}{c c c c c}
np1 & v & pp (with/to) & np2  & Correct\\ \hline
interpretations & present & None & music & Yes\\
roar & changed & to & growl & No\\
orders & ending & None & campaign & Yes\\
disaster & struck & None & discovery & No\\
book & went & to & press & No \\
man & sought & None & justice & Yes \\
church & meets & None & change & No\\
owners & met & None & defeat & No\\
pike & jumped & to & conclusion & No \\
determination & deserve & None & help & No\\
children & invited & to & concert & No\\
bureau & contributed & to & planning & Yes\\
work & demands & None & attention & Yes\\
corporation & making & None & acquisition& Yes \\
food & produces & None & changes & Yes\\
quest & offers & None & careers & Yes\\
total & marketing & None & program & No\\
distress & associated & with & performance & Yes \\
christianity & contributed & to & education & Yes\\
grizzlies & winning & None & streak & Yes\\
screen & drying & None & racks & No\\
husband & drops & None & case & Yes\\
company & developed & None & selection & Yes\\
president & gave & None & word & Yes\\
fatigues & made & None & streak & No\\
counterattack & takes & None & line & No\\
beep & turned & to & buzz & No\\
palace & issued & None & statement & Yes\\
president & took & None & pains & Yes\\
clubs & took & None & line & No\\
debate & led & to & decision & Yes\\
teacher & takes & None & hold & Yes\\
committee & submitted & None & report & Yes\\
cities & required & None & permits & No\\
body & held & None & fire & No\\
playing & gave & None & work & No\\
age & rewrites & None & events & No\\
\end{tabular} \\

\begin{tabular}{c c c c c}
np1 & v & pp (with/to) & np2  & Correct\\ \hline
issue & lay & with & president & No\\
group & conducting & None & review & Yes\\
10000 & claiming & None & event & No\\
gestapo & destroyed & None & rest & Yes\\
emotions & related & to & fight & Yes\\
play & running & None & bar & No\\
president & asks & None & support & Yes\\
ball & came & to & rest & No\\
man & made & None & reply & No\\
will & gives & None & way & Yes\\
means & make & None & provision & Yes\\
autumn & dulls & None & blaze & Yes\\
dilation & accompanying & None & elongation & No
\end{tabular} \\ \\ \\

From the above 50 examples, I believe that 26 are correct. In total there were 77 potential causal relations found. From these 68 were formed by the rule,\\ \\
\small
\textit{(!ABSTRACTION ,  * , EVENT or ACT)}\\ \\
\normalsize
One was formed by the rule,\\ \\
\small
\textit{(*,lead to,!ENTITY and !GROUP)}\\ \\ \normalsize which was, \textit{(debate ,led to, decision)}\\ \\
also one relation was formed by the rule,\\ \\ \small
\textit{(!ENTITY,associated with,!ABSTRACTION and !GROUP and !POSSESSION)}\\ \\ \normalsize which was, \textit{(distress, associated with, performance)}\\ \\
Some short comings of my program were its inability to use more complicated noun phrases as elements in the causal relation. This is because (as mentioned before) it is unclear where propositional phrases attach and which parts of the NP can be dropped in back-off while still keeping the relation in tact. \\
Also some of the patterns are very general and can include a lot of false-positive relations, this is shown by the large effect of the first rule mentioned with generates 68 out of the 77 potential relations.	

\end{document}
